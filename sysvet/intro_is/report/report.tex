\documentclass[12pt,a4paper,titlepage]{article}
%\documentclass[a4paper,12pt]{report}
%\usepackage[T1]{fontenc}
\usepackage[utf8]{inputenc}
\usepackage[swedish]{babel}
\usepackage{color}
\usepackage{graphics}

\title{\textsf{\textbf{Introduktion till informationssystem; \\ 
    En jämförande studie av: \\
    \textit{E-business} (Paul Beynon-Davies) och \\
    \textit{The Work System Method} (Steven Alter)}}}
\author{John-Patrik Nilsson \\
	e-post: daj01jni@student.lu.se}

\begin{document}

%\pagestyle{empty}
%\newpage

\pagestyle{headings}
\thispagestyle{empty}


%%%%%%%%%%%%%%%%%%%%%%%%%%%%%%%%%%%%%%%%%%
%%  Inledande del (tänk: läsarservice)  %%
%%%%%%%%%%%%%%%%%%%%%%%%%%%%%%%%%%%%%%%%%%

\maketitle

\section*{\textsf{Inledande sammandrag}}
%  Max 100 ord, bör innehålla: syfte, genomförande (metod, material,
%  resultat), resultat.
Det har varit en närmast explositionsartad utveckling av användandet och intresset för handel över Internet och andra elektroniska system under de senaste åren. Som grund för detta liggen en förhoppning av att introduceringen av informationsteknologibaserade informationssystem i verksamheter ska öka värdegenereringen och förbättra verksamhetens infrastruktur. Men för att kunna göra det behövs en analys och utvärdering av den verksamhet man vill förbättra samt en klar bild av vad informationsteknologi är kapabel att åstakomma.
\\
\\
\textbf{Nyckelord:} Informationssystem, system, informationsteknologi, elektronisk handel, Internet.

\tableofcontents
%  Både litteraturlistan samt det inledande sammandraget 
%  (abstract) ska vara med i innehållsförteckningen, men ingen av dem
%  ska ha ett kapitelnummer.

\section{\textsf{Förord}}
Denna rapport skrevs som en del av kursen "Introduktion till Informationssystem" vid Lunds Universitet, institutionen för informatik, hösten 2010. 

Tänkta läsare är såväl undervisare som studenter på nämnd kurs, men eftersom det är en introduktionskurs inom ämnet informatik har rapporten inga antaganden om förkunskap inom området, och kan därmed även vara till intresse för andra.

Initialt var det bestämt att skrivandet av rapporten skulle ske i en grupp om fyra personer, men p.g.a. diverse externa parametrar samt brister i kommunikationen i gruppen ändrades detta i sista minut och blev ett individuellt arbete.

Det bör understrykas att denna rapports enda syfte är att sammanfatta valda delar av två böcker, E-business (Beynon-Davies, 2004) och The Work System Method (Alter, 2006), med speciell inrikting på hur de ser på möjligheterna att förbättra existerande (affärs)verksamheter. Dessa två böcker är även skrivna med olika grad av fördjupning --- Beynon-Davies skriver om specifikt system inriktade på elektronisk handel medan Alter har en mer övergripande syn på system och dess utveckling.

\section{\textsf{Definitioner}}
Ämnet informatik --- liksom en klar majoritet av andra ämnen och områden --- har ett områdes-specifikt internt vokabulär av förkortningar,icke-standardiserade och icke-definerade begrepp och uttryck. På grund av detta kommer de begrepp och uttryck som förekommer i rapporten deklareras och defineras (så som de kommer att användas i rapporten) nedan på ett klart och tydligt sätt.
\\
\\
\textbf{System:} Ett ordlexikon på Internet (TheFreeDictionary.com, 2010) definerar ett system som en grupp av interagerande och sammankopplade eller icke-självständiga element som tillsammans bildar en komplex helhet. Denna skildring av system kan även hittas i boken E-business (Beynon-Davies, 2004).
\\
\\
\textbf{Informationssystem (IS):} Ett begrepp som används frekvent, men tolkas och defineras generellt alltid olika, beroende på vilka personer eller grupper som är inblandade. Denna rapport gör ingen egen definition av begreppet IS, utan drar slutsatsen att begreppet noga bör undersökas av läsaren utifrån det aktuella sammanhanget. Det bör dock understrykas att IS som koncept inte är beroende av informationsteknologi (se ICT nedan).
\\
\\
\textbf{Information and communication technology (ICT):} (Sv. Informations- och kommunikationsteknologi.) En term som används för att beskriva och sammanfatta de teknologier som används för att transportera information och data. I rapport används ICT specifikt som en benämning på de datordrivna system som behandlar information och data.
\\
\\
\textbf{E-handel:} Är en förkortning av "elektronisk handel" och beskriver den handelsverksamhet som bedrivs över elektroniska system. Exempel på sådana elektroniska system är Internet, men det inkluderar även andra datorbaserade nätverk.
%\\
%\\
%\textbf{Human activity system (HAS):} (Sv. Mänskligt aktivitetssystem, även kallat verksamhetsprocess.) Ett begrepp som beskriver de mänskliga  processerna och aktiviteterna i en del som besitter en nyckelposition inom en organisation eller verksamhet. (Beynon-Davies, 2004)
%\\
%\\
%\textbf{Work system:} 


%%%%%%%%%%%%%%%%
%%  Huvuddel  %%
%%%%%%%%%%%%%%%%

%  Är tre-delad.
%  Bör innehålla: syfte, bakgrund/forskningsöversikt, material,
%  metod, resultat, diskussion/kommentar, slutsats/värdering/summering.

\section{\textsf{Inledning}}
\subsection{\textsf{Verksamheter, handel och informationsteknologi}}
%IS ICT E-handel
%utveckling de senaste/kommande åren

Under det senaste decenniet har ökningen av verksamheter som bedriver elektronisk handel, eller på annat sätt använder datorbaserad informationsteknologi varit explosionsartad. Denna trend är global och ser ungefär likadan ut för alla i-länder. % statistik.

I en undersökning av Forrester Research (Forrester Research Web-Influenced Retail Sales Forecast, 12/09 (US)) kommer elektronisk handel i USA fortsätta att öka med 10\% fram till år 2014, och att elektronisk handel kommer att stå för 50\% av den totala handeln år 2012.

Det finns förmodligen ett flertal anledningar till detta\footnote{vissa är utom denna rapports räckvidd.}, men en av de största kan antas vara att den globalisering som Internet och nätverk ligger till grund för öppnar en betydligt större marknad för potentiella kunder. En annan fördelaktig produkt av denna globalisering och elektroniska utveckling av verksamheter är att det blir möjligt att decentralisera både verksamheten samt delar av verksamheten för att göra det mer ekonomiskt fördelaktigt. Som ett exempel skulle man kunna minska hyres- och lagerkostnader genom att etablera kontor och lager där det kostar minst samtidigt som man håller kundkontakten och distributionskanalerna öppna med hjälp av informations- och kommunikationsteknologi.

En annan anledning till att verksamheter ökat sin närvaro via de elektroniska kanalerna skulle kunna vara ett försök till att effektivisera och rationalisera de inre strukturer som finns i verksamheten. Det skulle i vissa fall kunna representas i kortare beslutsvägar, sänkta strukturella kostnader samt ett ökat värde av slutprodukten.

\subsection{\textsf{Verksamheten som ett system}}
För att få en övergripande och utvärderingsbar bild av hur en verksamhet eller en del av en verksamhet fungerar tillgriper både Paul Beynon-Davies i sin E-business såväl som Steven Alter i sin The Work System Method ett synsätt där de sammanfattar verksamheten som ett system. Dock skiljer sig de båda författarna sig åt när det gäller såväl namnet på systemet som den exakta definitionen av systemet.

Enligt Beynon-Davies, 2004, är det ett generellt antaget synsätt inom informatiklitteraruren att se på verksamheter som system.

Syftet med att se på verksamheter som system kan man tolka som ett försök att identifiera och generalisera de element som en verksamhet är uppbyggd av. Genom att generalisera systemet kan man sedan applicera generalistiska metoder för att utvecka och förbättra systemet (verksamheten) och dess delar.

En viktig anledning till varför man brukar se på verksamheter som system är även att det blir möjligt att utvärdera verksamhetens presetation och effektivitet.

Det kan tolkas som om att både The Work System Method (Alter, 2006)) samt E-business (Beynon-Davies, 2004) förklarar metoder och synsätt för att förklara, kvantisera, utvärdera verksamheten som ett system. Utifrån  detta argumenterar de sedan att det är möjligt att utveckla förbättringar av verksamheten.

\subsection{\textsf{Informationssystem}}
Informationssytem kan ses som den process som bedrivs för att kommunicera och transportera information mellan parter i ett system. Per definition behöver inte informationssystem någon typ av informationsteknologi för att fungera --- det fanns informationssystem i alla verksamheter redan innan begreppen elektronisk handel och informationsteknologi introducerades. Dock bör det tilläggas att informationsteknologi har gjort det möjligt att förbättra informationssystemen i alla aspekter, vilket är en anledning till att de båda begreppen ofta används synonymt.

% \subsection{\textsf{Informationssystem och informationsteknologi}}
% Verksamhet och IT?
% Hur används det inom företag/verksamhet?

Enligt Beynon-Davies (2004) har man under det senaste decenniet kunnat se en tydlig trend att de ekonomiska marknaderna blir mer och mer inriktade på användningen av elektroniska informationssystem (IS).

Vidare har samtliga av de största globala företagen (listade i \textit{Fortune} Global 500) etablerat en stark närvaro på Internet --- en ökning från 50\% sedan 1997 (Beynon-Davies, 2004).

Syftet med denna rapport är att presentera en jämförande studie av hur två författare --- Alter och Beynon-Davies --- ser på möjligheten att förbättra en verksamhet med hjälp av informationssystem (IS).

Som primära underlag för studien har två böcker använts; \textit{E-business} av Paul Beynon-Davies (Beynon-Davies, 2004), samt \textit{The Work System Method} av Steven Alter (Alter, 2006).

\section{\textsf{Förbättringsmetoder för system}}
\subsection{\textsf{Vad menas med förbättring av ett system?}}
och hur beräknar man det

I boken E-business (2004) menar Beynon-Davies att ett system kan förklaras och sammanfattas som ett system. Han ger en klar bild av en definition av denna typ av system, samt en summering av dess delar. Han argumenterar även att ett system inte kan studeras utifrån dess inkluderande delar, utan måste utvärderas utifrån en holistiskt perspektiv.

Utifrån detta poängterar han även att ett system --- per definition --- har beräkningsbara parametrar vad gäller effektivitet och prestanda inom alla dess aspekter och hela systemets (verksamhetens) värdekedja.

%\subsection{\textsf{Förbättring av delar eller hela systemet?}}
%\subsection{\textsf{Metoder för förbättring av systemet? --- Hur?}}
\subsection{\textsf{Utveckling och utvärdering av system}}
Utifrån systemets aspekter gällande effektivitet och prestanda menar Beynon-Davies (E-business, 2004) att systemet kan beräknas och utvärderas. Dock beskriver författaren inga konkreta metoder eller tekniker för att förbättra ett system, istället ger han riktlinjer och metoder för hur ICT-baserade informationssystem utvecklas bör utvecklas för att uppfylla krav på effektivitet och prestanda.

I boken Work System Method (Alter, 2006) ges däremot konkreta listor, tabeller och metoder för hur ett system\footnote{Alter använder sitt begrepp \textit{Work System} för att beskriva verksamhetssystem.} bör utvärderas och förbättras. Men, Alter skriver även att man bör etablera ett synsätt där allt i analysen av ett system samt dess krav förmodligen kommer ändras under utvecklingens gång. Detta är även en grundpelare för programmeringsmetoder som bygger på \textit{Agila (En. Agile)} metoder\footnote{Extreme Programming (XP) bygger på agila metoder.} (Martin, 2003).

\section{\textsf{Slutsats och kommentarer}}
Vid en första anblick kan det tyckas att Beynon-Davies bok E-business är väl strukturerad och tar upp metoder och synsätt vilket underlättar för utveckling och förbättring av system. Så må vara fallet, men det är denna rapports slutsats att bokens detaljrikedom är till nackdel för Beynon-Davies förmåga att kommunicera ett klart och tydligt budskap. Det kan även argumenteras att E-business fokusering på vilka delar ett system är uppbyggt av gör att de andra aspekterna av systemutveckling i boken får för lite utrymme. 

Dessutom kan det vara en nackdel att boken E-business är så detaljerad och fokuserad på just elektronisk handel och utveckling av system som behandlar det, när teknologin som sådan handel vilar på utvecklas och förändras i stort sett dagligen.

Alter i sin bok The Work System Method har däremot antagit ett mer generellt synsätt på utveckling av verksamhetssystem, och där han även poängterar att man bör inte lägga för stor fokus på elektroniska och ICT-baserade system för att förbättra en verksamhet. Det är denna rapports slutsats att dett är en bra och klok inställning.


%%%%%%%%%%%%%%%%%%%%%%
%%  Avslutande del  %%
%%%%%%%%%%%%%%%%%%%%%%

\section*{\textsf{Litteratur}}
Beynon-Davies, Paul, 2004: \textit{E-Business}. 
Palgrave Macmillan. 
ISBN: 978-1-4039-1348-7.
\\
\\
Alter, Steven, 2006: \textit{The Work System Method: Connecting People, Processes, and IT for Business Results}.
ISBN: 0-9778497-0-8.
\\
\\
Martin, Robert C., 2003: \textit{Agile Software Development}.
Pearson Education, Inc.
ISBN: 0-13-597444-5.
\\
\\
TheFreeDictionary.com, 2010. \textit{http://www.thefreedictionary.com/system}
\\
\\
Wikipedia.org, 2010. \textit{http://en.wikipedia.org/wiki/Information\_system}

\appendix
%\section{\textsf{Bilagor}}

%\section{\textsf{Register}}
\end{document}
