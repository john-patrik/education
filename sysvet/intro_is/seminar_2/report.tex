\documentclass[a4paper,10pt]{article}
%\usepackage[T1]{fontenc}
\usepackage[utf8]{inputenc}
%\usepackage[swedish]{babel}
\usepackage{color}
\usepackage{graphics}

\title{Introduktion till informationssystem \\
	Akademiskt skrivande \\
    Seminarium 2 \\
    Review of E-Business (Beynon-Davies, 2004), Part 1}
\author{John-Patrik Nilsson \\
	e-mail: daj01jni@student.lu.se}

\begin{document}

\maketitle
%\tableofcontents

\pagestyle{empty}
\thispagestyle{empty}

\section{Abstract}
The understanding of information systems (IS) and human activity systems (HAS) is of great importance when it comes to implement information and communication technology systems (ICT) to support businesses.

\textbf{Keywords:} Information systems, IS, information and communication technology, ICT, business, e-business, i-business.

\section{E-Business}
The definition of the term e-business is generally thought of as a grouping or set of activities in a commercial organisation which are fully supported by (and relies on) information and communications technology.

\section{Systems}
The definition of a system could be explained as an organised collection of interdependent parts which works together to reach a common goal which neither of them could have reached without the other parts.

By abstracting organisations and their parts as systems it becomes possible to measure the effectiveness and value of those. So too is it with information systems which use information and communications technology.

A system typically resides in an \textit{environment} where it recieves \textit{input}, \textit{processes} it, and produces \textit{output}. This is the process of a system.

Furthermore, systems can be divided into soft and hard systems. A human activity system is typically a soft system; it includes humans and as such the system is not exact in all its definitions and outcomes.

\appendix
\section{References}
Beynon-Davies, Paul. \textit{E-Business, (2004)}, Palgrave Macmillan. ISBN: 978-1-4039-1348-7.

\end{document}
