\documentclass[a4paper,12pt]{article}
%\usepackage[T1]{fontenc}
\usepackage[utf8]{inputenc}
\usepackage[swedish]{babel}
\usepackage{color}
\usepackage{graphics}


\title{\textsf{\textbf{SYSA01: Mjukvaruutveckling \\
    inlämningsuppgifter 1--4}}}

\author{John-Patrik Nilsson \\
    820610-4070 \\
	e-post: daj01jni@student.lu.se}

\begin{document}

%\pagestyle{empty}
\thispagestyle{empty}

\maketitle
\newpage

\tableofcontents

\newpage
\section{\textsf{Inlämningsuppgift 1}}
\subsection{\textsf{Del a}}
\begin{verbatim}
import java.io.*;
import java.util.ArrayList;

class A {
    public static void main(String[] args) 
    {
        ArrayList<String> input_strings = new ArrayList<String>();
        InputStreamReader input = new InputStreamReader(System.in);
        BufferedReader reader = new BufferedReader(input);
        int nbr_of_words = 2;
        String string = "";

        while (input_strings.size() < nbr_of_words)
        {
            System.out.print("Type a word: ");

            string = "";
            try
            {
                string = reader.readLine();
            }
            catch(Exception e) {}
    
            if (string != "")
            {
                // System.out.println("You typed: " + string);
                input_strings.add(string);
            }
        }

        string = "";
        System.out.print("The concatenated string is: ");
        for (String s : input_strings)
        {
            // System.out.print(s + " ");
            string += s + " ";
        }
        string = string.trim();
        System.out.print(string);
        System.out.println(".");
        System.out.println(string.length());
    }
}
\end{verbatim}

\newpage
\subsection{\textsf{Del b}}
\begin{verbatim}
import java.io.*;
import java.util.Scanner;

class B {
    public static void main(String[] args) 
    {
        // We need this for the input-mechanic.
        InputStreamReader input = new InputStreamReader(System.in);
        BufferedReader reader = new BufferedReader(input);

        // This is the needed user input.
        String name = "";
        double salary = 0;
        int h = 0;


        // While we do not have all the info we need,
        // continue to ask for it.
        while (name == "")
        {
            System.out.println("Ange ditt namn: ");
            try
            {
                name = reader.readLine();
                name = name.toUpperCase();
            }
            catch (Exception e) {}
        }

        Scanner in_scanner = new Scanner(System.in);

        while (salary == 0)
        {
            System.out.println("Ange din timlön: ");
            try
            {
                //tmp_string = reader.readLine();
                //salary = Double.parseDouble(tmp_string);
                salary = in_scanner.nextDouble();
            }
            catch (Exception e) {}
        }

        while (h == 0)
        {
            System.out.println("Ange dina arbetade timmar: ");
            try
            {
                h = in_scanner.nextInt();
            }
            catch (Exception e) {}
        }

        in_scanner.close();

        // Formatting crap.
        String out_string = name + " du tjänade " + salary*h + 
            " kr förra veckan";
        out_string = out_string.replace(".", ",") + ".";

        System.out.println(out_string);
    }
}
\end{verbatim}


\newpage
\section{\textsf{Inlämningsuppgift 2}}
\begin{verbatim}
import java.io.*;
import java.util.Scanner;

class A {

    public static void main(String[] args)
    {
        // Get user input.
        int a = 0;
        int b = 0;


        Scanner in = new Scanner(System.in);

        while (true)
        {
            System.out.print("Ange det första nummret: ");
            try
            {
                a = in.nextInt();
            }
            catch (Exception e) {}

            if (a == 0)
            {
                break;
            }

            System.out.print("Ange det andra nummret: ");
            try
            {
                b = in.nextInt();
            }
            catch (Exception e) {}

            // Compare values.
            if (a > b)
            {
                System.out.println(a +" är större än "+ b);
            }
            else if (b > a)
            {
                System.out.println(b +" är större än "+ a);
            }
            else if (a == b)
            {
                System.out.println("Talen är lika stora.");
            }
        }

        System.out.println("Tackar.");
        in.close();
    }
}
\end{verbatim}


\newpage
\section{\textsf{Inlämningsuppgift 3}}
\begin{verbatim}
import java.io.*;
import java.util.Scanner;
import java.util.ArrayList;

class A {

    public static void main(String[] args)
    {
        ArrayList<Integer> list = new ArrayList<Integer>(5);
        Scanner in = new Scanner(System.in);
        int largest = 0;
        double mean = 0;
        int sum = 0;
        int current;    // We don't want to access the list more 
                        // than we need.

        System.out.println(
            "Type in 5 numbers, each followed by <enter>:");

        for (int i = 0; i < 5; i++)
        {
            try
            {
                list.add(in.nextInt());
            }
            catch (Exception e) {}
        }

        for (int i = 0; i < list.size(); i++)
        {
            current = list.get(i);
            // System.out.println(current);

            // Find largest nbr:
            if (largest < current)
            {
                largest = current;
            }

            // Calculate sum:
            sum += current;
        }

        // Calculate mean:
        mean = (double) sum / (double) list.size();

        System.out.println("Largest number: " + largest);
        System.out.println("Sum: " + sum);
        System.out.println("Mean value: " + mean);
    }
}
\end{verbatim}


\newpage
\section{\textsf{Inlämningsuppgift 4}}
\subsection{\textsf{testBankAccount.java}}
\begin{verbatim}
import java.io.*;

class TestBankAccount {
    
    public static void main(String[] args)
    {
        String id = "meow_01";
        BankAccount account = new BankAccount(id);
        String category;


        // Test initial balance.
        category = "Initial balance";

        if (account.getBalance() == 0)
        {
            System.out.println(category + " is correct!");
        }


        // Test id.
        category = "id";

        if (account.getId().equals(id))
        {
            System.out.println(category + " is correct!");
        }


        // Test deposit.
        category = "Deposit function";

        if (account.deposit(100))
        {
            if (account.getBalance() == 100)
            {
                System.out.println(category + " is correct!");
            }
        }


        // Test legal withdraw.
        category = "Legal withdraw function";

        if (account.withdraw(account.getBalance()))
        {
            if (account.getBalance() == 0)
            {
                System.out.println(category + " is correct!");
            }
        }


        // Test illegal withdraw.
        category = "Illegal withdraw function";

        if (account.withdraw(account.getBalance()+1) == false)
        {
            System.out.println(category + " is correct!");
        }
    }
}
\end{verbatim}

\newpage
\subsection{\textsf{bankAccount.java}}
\begin{verbatim}
class BankAccount {

private String id;
private int balance;

    public BankAccount(String id)
    {
        this.id = id;
        this.balance = 0;
    }

    public BankAccount(String id, int balance)
    {
        this.id = id;
        this.balance = balance;
    }

    public String getId()
    {
        return id;
    }

    public int getBalance()
    {
        return balance;
    }

    // I know this should return void, but boolean is more useful.
    public boolean deposit(int amount)
    {
        balance += amount;
        return true;
    }

    public boolean withdraw(int amount)
    {
        if (balance >= amount)
        {
            balance -= amount;
            return true;
        }
        else
        {
            // Insufficient funds.
            return false;
        }
    }
}
\end{verbatim}

\newpage
\subsection{\textsf{atm.java}}
\begin{verbatim}
import java.io.*;
import java.util.Scanner;

class Atm {
    
    public static void main(String[] args)
    {
        BankAccount account = new BankAccount("meow_02");
        Scanner in = new Scanner(System.in);
        boolean again = true;
        boolean deposit, withdraw;
        int amount;

        while (again)
        {
            deposit = false;
            withdraw = false;

            System.out.print("Press 0 to deposit or 1 to withdraw: ");
            try 
            {
                int tmp = in.nextInt();
                if (tmp == 0) // This means deposit.
                {
                    deposit = true;
                    withdraw = false;
                }
                else if (tmp == 1) // This means withdraw.
                {
                    deposit = false;
                    withdraw = true;
                }
                else // This means the user is dumb.
                {
                    deposit = false;
                    withdraw = false;
                }
            }
            catch (Exception e) {}

            // If the user doesn't want anything.
            if (!deposit && !withdraw)             
            {
                // continue;

                System.out.println("Bye!");
                //again = false;
                break;
            }

            System.out.print("Amount: ");
            try
            {
                amount = in.nextInt();
                if (deposit)
                {
                    account.deposit(amount);
                }
                else if (withdraw)
                {
                    account.withdraw(amount);
                }
            }
            catch (Exception e) {}

            System.out.println("Current balance: "+account.getBalance());
            System.out.print("Do you want to play again? (y/n): ");
            try
            {
                String ans = in.next();
                if (ans.equals("n"))
                {
                    again = false;
                }
            }
            catch (Exception e) {}
        }
    }
}
\end{verbatim}

\end{document}
