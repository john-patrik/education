\documentclass[a4paper,12pt]{article}
%\usepackage[T1]{fontenc}
\usepackage[utf8]{inputenc}
\usepackage[swedish]{babel}
\usepackage{color}
\usepackage{graphics}


\title{\textsf{\textbf{Introduktion till informationssystem; \\ 
    En jämförande studie av Beynon-Davies (E-business) och
    Alter (The Work System Method)}}}
\author{John-Patrik Nilsson \\
	e-post: daj01jni@student.lu.se}

\begin{document}

%\pagestyle{empty}
%\newpage
%\thispagestyle{empty}


%%%%%%%%%%%%%%%%%%%%%%%%%%%%%%%%%%%%%%%%%%
%%  Inledande del (tänk: läsarservice)  %%
%%%%%%%%%%%%%%%%%%%%%%%%%%%%%%%%%%%%%%%%%%

\maketitle

\section*{\textsf{Inledande sammandrag}}
%  Max 100 ord, bör innehålla: syfte, genomförande (metod, material,
%  resultat), resultat.

\textbf{Nyckelord:} 

\tableofcontents
%  Både litteraturlistan samt det inledande sammandraget 
%  (abstract) ska vara med i innehållsförteckningen, men ingen av dem
%  ska ha ett kapitelnummer.

\section{\textsf{Förord}}

\subsection{\textsf{Tack}}
Till handledare och medhjälpare.


%%%%%%%%%%%%%%%%
%%  Huvuddel  %%
%%%%%%%%%%%%%%%%

%  Är tre-delad.
%  Bör innehålla: syfte, bakgrund/forskningsöversikt, material,
%  metod, resultat, diskussion/kommentar, slutsats/värdering/summering.

\section{\textsf{Inledning}}

\section{\textsf{Huvudtext}}

\section{\textsf{Avslutning}}


%%%%%%%%%%%%%%%%%%%%%%
%%  Avslutande del  %%
%%%%%%%%%%%%%%%%%%%%%%

\section*{\textsf{Litteraturförteckning}}
Beynon-Davies, Paul. \textit{E-Business, (2004)}, Palgrave Macmillan. ISBN: 978-1-4039-1348-7.

\appendix
\section{\textsf{Bilagor}}

\section{\textsf{Register}}
\end{document}
