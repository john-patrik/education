\documentclass[a4paper,12pt]{article}
%\usepackage[T1]{fontenc}
\usepackage[utf8]{inputenc}
\usepackage[swedish]{babel}
\usepackage{color}
\usepackage{graphics}


\title{\textsf{\textbf{SYSA01: Systemanalys \& modellering \\
    inlämningsuppgifter}}}

\author{John-Patrik Nilsson \\
    820610-4070 \\
	e-post: daj01jni@student.lu.se}

\begin{document}

%\pagestyle{empty}
\maketitle
\thispagestyle{empty}

\newpage
\tableofcontents

\newpage
\section{\textsf{Introduktion}}
\subsection{\textsf{Definitioner}}
\begin{description}
\item[Probleområde:] Systemets syfte, innehåller bl.a. händelsetabeller och klassdiagram. Kan även beskrivas som systemets funktionella krav. Förklarar vad systemet gör.
\item[Användningsområde:] Beskriver hur systemets beteende och hur det agerar för att uppnå sitt syfte. Dokumentation av användningsområde innehåller ofta aktörertabeller, användningsfall, aktördefinitioner, gränssnitt och funktionslistor.
\item[Aktörer:] Defineras som någon eller något som inte tillhör målsystemet men på något sätt interagerar med det. Aktörer kan även vara andra system.
\item[Användningsfall:] Beskriver interktionen mellan systemet och dess aktörer.
\item[Systemdefinition:] 
\item[VATOFA-kriteriet:] Är en 
\end{description}

\subsection{\textsf{Användningsfall}}
Ett smidigt sätt att utforma användningsfall är att använda sig av följande process:

\begin{enumerate}
\item Identifiera systemets aktörer.
\item Definera och beskriv aktörernas interaktion med systemet. Aktörernas olika arbetsuppgifter där de använder systemet är ett användningsfall.
\end{enumerate}

Användningsfall kan även beskrivas med hjälp av ett tillståndsdiagram.

\subsection{\textsf{Aktörspecifikation}}
En aktörspecifikation består av följande delar:

\begin{description}
\item[Mål:] Vilka mål aktören förväntas uppnå genom att använda systemet.
\item[Kännetecken:] Vilka attribut (t.ex. erfarenhet) aktören förväntas ha.
\item[Exempel:] Beskriv ett exempel då aktören interagerar med systemet.
\end{description}

\newpage
\section{\textsf{Systemval}}

\newpage
\section{\textsf{Modellering 1}}

\newpage
\section{\textsf{Modellering 2}}

\newpage
\section{\textsf{Användningsområde}}

\end{document}
