\documentclass[a4paper,10pt]{article}
%\usepackage[T1]{fontenc}
\usepackage[utf8]{inputenc}
\usepackage[swedish]{babel}
\usepackage{color}
\usepackage{graphics}

\title{Linux development environment \\
	Laboration 1}
\author{John-Patrik Nilsson \\
	e-mail: jpatrik.nilsson@gmail.com \\
	Skype: j-p.nilsson}

\begin{document}

\maketitle
%\tableofcontents

\pagestyle{empty}
\thispagestyle{empty}

\section{Operating systems}
An operating system is software which governs the usage of both hardware and software resources of a computerized system, such tasks includes, as an example, managing processor, memory disk space, etc.. Essentially an operating system acts as a host to the various software applications on the system. The operating system, often abbreviated as 'OS', also serves as an interface through which other software applications can communicate with the hardware. Because of this, the software applications which resides and operates on the system can function without extensive knowledge of the hardware of the system. 

The central and most important part of an operating system is the kernel.

\section{Kernels}
All but the most trivial of computer systems today have what is called a kernel. The kernel is the innermost abstraction layer of an operating system, it is the last line of software before the hardware. Important tasks which a kernel usually is assigned to is scheduling, memory government, and inter-process communication (communitation between the various processes which currently runs on the system).

\section{Linux}
Linux is a Unix-like operating system which is based on the Linux kernel, created by Linus Torvalds who at the time of writing this document is still directing the development of the kernel.

However, for Linux to be classified as a functional and usable operating system there would need to be a whole suite of software applications. This was to be provided at first by Richard Stallman who started the GNU Project with the intent of creating a complete Unix-compatible software system.

Due to the nature of the GNU public license and of the Unix project in general a multitude of distributions and variations of Unix and Linux systems have erupted. These have in common that they are all Unix-like products, but differ in many ways, some easily discernable and others not so easily discernable.

The many variations of Linux sortware suites is called Linux distributions, which each of the contains a unique and varying amount of software applications and modules. The main difference between the Linux distributions is user interaction, whilst the core functionality remains the same in all distributions.

\section*{References}
\textit{http://en.wikipedia.org/wiki/Operating\_system} \\
\textit{http://en.wikipedia.org/wiki/Kernel\_(computer\_science)} \\
\textit{http://en.wikipedia.org/wiki/Linux\#Current\_development} \\
\textit{http://computer.howstuffworks.com/operating-system2.htm} \\

\end{document}
