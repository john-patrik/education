\documentclass[a4paper,10pt]{article}
%\usepackage[T1]{fontenc}
\usepackage[utf8]{inputenc}
\usepackage[swedish]{babel}

\title{Övning 1}
\author{John-Patrik Nilsson}

\begin{document}

\maketitle

\pagestyle{empty}
\thispagestyle{empty}

\section{Del A}
Utan någon som helst vetenskaplig grund att stå på vad gäller denna fråga tror jag att en webbaserad kurs skiljer sig från en vanlig, traditionell, universitets- eller högskolekurs på flera olika sätt. Främst blir det skillnad vad gäller assimilation, behandling, och slutligen redovisning av den information och kunskap som studenten ska ha förvärvat sig. Genom att dela upp kursen i flera deluppgifter och inlämningar, samt att utforma uppgifterna på ett sådant sätt att det krävs en djupare förståelse om problematiken för att kunna redovisa dem, kan kursansvarig försäkra sig om att deltagarna tillgodogör sig en godtycklig mängd kunskap. Kort sagt resulterar ofta webbaserade kurser i att deltagarna producerar mer material totalt sett; till inlämningar, övningar, laborationer, etc..

En annan följd av att kursen hålls via Internet blir att all inhämntning av information blir visuell (inga föreläsningar, endast text), men det behöver ju faktiskt inte vara så, man skulle kunna producera föreläsningar och distribuera dem via webbsidan.

Den sociala skillnaden blir, tror jag, att studenter inte umgås utanför "klassrummen" i en webbkurs; det blir inget prat om annat än just det aktuella ämnet. Det blir på ett sätt mer proffessionellt, men man kan begrunda värdet av den sociala bit man går miste om. En fördel skulle kunna vara att studenterna blir mer villiga att ställa frågor som ett resultat av den högre grad av anonymitet som Internet tillhandahåller.

\newpage
\section{Del B}
Jag är inte helt ny inom Linuxvärlden, och jag har viss kunskap om programmering, men det är fortfarande en hel del ämnen i denna kursen som är främmande för mig.

Efter en ytlig granskning av innehållet på kursen planerar jag att behöva runt 170 timmar för att klara av kursen. 

Här följer min bedömning av hur många timmar respektive kapitel tar att slutföra:
\\

\begin{tabular}{|cc|} \hline
 \emph{kapitel} & \emph{timmar} \\ \hline \hline
 1 & -- 	\\	\hline
 2 & 18 	\\	\hline
 3 & 6 		\\	\hline
 4 & 14 	\\	\hline
 5 & 10		\\	\hline
 6 & 14		\\	\hline
 7 & 10		\\	\hline
 8 & --		\\	\hline
 9 & 24		\\	\hline
 10 & 18 	\\	\hline
 11 & 16 	\\	\hline
 12 & 12 	\\	\hline
 13 & 16 	\\	\hline
 14 & -- 	\\	\hline
 15 & -- 	\\	\hline
\end{tabular}
\\

Utefter min bedömning i föregående tabell har jag gjort en preliminär planering där jag deklarerar vilken vecka jag planerar att lämna in respektive uppgift.
\\

\begin{tabular}{|cc|} \hline
 \emph{kapitel} & \emph{inlämn., v.} \\ \hline \hline
 2 & 4		\\ 	\hline
 3 & 4		\\ 	\hline
 4 & 5		\\ 	\hline
 5 & 6		\\ 	\hline
 6 & 8		\\ 	\hline
 7 & 10		\\ 	\hline
 8 & --		\\ 	\hline
 9 & 13		\\ 	\hline
 10 & 15	\\ 	\hline
 11 & 17	\\ 	\hline
 12 & 19	\\ 	\hline
 13 & 21	\\ 	\hline
\end{tabular}

\end{document}
